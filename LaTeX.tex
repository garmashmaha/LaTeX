\documentclass{article}
\usepackage[utf8]{inputenc}
\usepackage[T1]{fontenc}
\usepackage{indentfirst}
\usepackage{microtype}
\microtypesetup{protrusion=true}
\usepackage[polish]{babel}
\usepackage{url}
\usepackage{hyperref}
\usepackage{geometry}
\geometry{a4paper, margin=1in}

\title{Algorytmy rekomendacyjne i ich zastosowania}
\author{Autor: Mariia Harmash}
\date{Kwiecień 2024}

\begin{document}

\maketitle
\newpage

\section{Wstęp}

Algorytmy rekomendacyjne są nieodłącznym elementem naszego codziennego życia i kluczowym elementem wielu platform internetowych,
od sklepów internetowych po serwisy strumieniowania muzyki i filmów. Są one wykorzystywane do personalizacji doświadczeń użytkowników
poprzez sugerowanie produktów lub treści, które mogą ich zainteresować.

\section{Omówienie problemu}

Algorytmy rekomendacyjne rozwiązują wiele problemów zarówno dla użytkowników, jak i dla firm. \newline
Po pierwsze, umożliwiają spersonalizowane doświadczenia online. Dzięki nim, serwisy internetowe, takie jak Spotify, są w stanie dostosować swoje propozycje 
do indywidualnych preferencji użytkowników. Na przykład, jeśli jesteś fanem jazzu, algorytm Spotify automatycznie będzie proponować Ci więcej utworów tego gatunku.\newline
Po drugie, algorytmy te pomagają w optymalizacji czasu. Zamiast przeszukiwania setek opcji, użytkownicy mogą skorzystać z preselekcji, którą przygotował dla nich algorytm. 
Dzięki temu, oszczędzają czas, który mogliby spędzić na przeglądaniu nieodpowiednich dla nich treści.\newline
Po trzecie, dla firm, algorytmy rekomendacyjne są kluczowe w zwiększaniu zaangażowania i lojalności klientów. Poprzez dostarczanie użytkownikom treści, 
które są dla nich najbardziej interesujące, firmy są w stanie zbudować silniejszą relację z klientami. W dłuższej perspektywie, przekłada się to na wyższe zyski.

\section{Analiza koszykowa}

Analiza koszykowa, znana również jako analiza sekwencji zakupów lub reguły asocjacyjne, to technika używana w dziedzinie data mining, która pozwala na odkrywanie wzorców zakupowych 
klientów poprzez analizę dużych zbiorów danych transakcyjnych. Jest to metoda szczególnie przydatna w handlu detalicznym i e-commerce, gdzie pomaga w identyfikacji produktów często 
kupowanych razem, co może przyczynić się do optymalizacji strategii marketingowych, asortymentu produktowego oraz promocji. Wykorzystuje algorytmy takie jak Apriori, FP-Growth 
i Eclat do przeszukiwania danych i identyfikacji częstych zestawów produktów. Na podstawie tych zestawów formułuje reguły asocjacyjne, które wskazują na prawdopodobieństwo zakupu 
jednego produktu przy zakupie innego. \newline
Dzięki analizie koszykowej firmy mogą lepiej zarządzać zapasami, tworzyć skuteczne promocje krzyżowe, personalizować oferty, optymalizować układ sklepu i projektować kampanie 
reklamowe, co przekłada się na wzrost sprzedaży i satysfakcji klientów. Ponadto, analiza ta może być wykorzystywana w różnych sektorach, takich jak bankowość czy opieka zdrowotna, 
do identyfikacji wzorców zachowań wskazujących na ryzyko oszustwa lub potrzeby zdrowotne określonych grup pacjentów.

\section{Zastosowanie w muzyce cyfrowej i grach komputerowych}

Algorytmy rekomendacji odgrywają kluczową rolę w muzyce cyfrowej i grach komputerowych. Są one wykorzystywane do dostosowywania treści do indywidualnych preferencji użytkowników, 
co pozwala na personalizację doświadczeń. Dzięki temu użytkownicy mogą odkrywać nowe utwory muzyczne i gry, które mogą im się spodobać, na podstawie ich dotychczasowych wyborów 
i zachowań.\newline
Poprzez rekomendowanie interesujących treści, algorytmy rekomendacji przyczyniają się do zwiększenia zaangażowania użytkowników. To z kolei prowadzi do większej ilości czasu 
spędzanego na platformach muzycznych i gamingowych.
Dodatkowo, rekomendacje mogą skutkować zwiększeniem sprzedaży muzyki i gier. Dzieje się tak, ponieważ algorytmy są w stanie sugerować produkty, które najprawdopodobniej przyciągną 
uwagę konkretnego użytkownika. W ten sposób algorytmy rekomendacji przyczyniają się do optymalizacji sprzedaży na platformach muzycznych i gamingowych.

\section{Popularne algorytmy rekomendacyjne}

Algorytmy rekomendacyjne odgrywają kluczową rolę w dzisiejszym Internecie, pomagając odkrywać nowe treści i produkty, które mogą nas zainteresować. Jednym z najpopularniejszych 
rodzai takich algorytmów jest filtracja kolaboratywna, która analizuje wzorce zachowań grupy użytkowników, aby na ich podstawie przewidywać preferencje indywidualnego użytkownika. 
Innym powszechnie stosowanym algorytmem jest filtracja treściowa, która skupia się na analizie charakterystyk produktu i dopasowaniu ich do profilu użytkownika.\newline
Wiele platform internetowych korzysta również z hybrydowych podejść, które łączą różne typy algorytmów, takie jak historia przeglądania użytkownika i semantyczna analiza treści, 
dla bardziej precyzyjnych rekomendacji. W ostatnich latach coraz bardziej popularne stają się zaawansowane techniki AI, które wykorzystują uczenie maszynowe do tworzenia coraz 
bardziej precyzyjnych i efektywnych rekomendacji, które mogą rozumieć kontekst i nastrój użytkownika.
Te algorytmy rekomendacyjne mają szeroki zakres zastosowań, od mediów po medycynę i produkcję, i ich wpływ na nasze życie społeczne i indywidualne jest coraz bardziej zauważalny. 

\section{Złożoność obliczeniowa}

Złożoność obliczeniowa algorytmów rekomendacji jest kluczowym zagadnieniem, które pozwala określić ilość operacji potrzebnych do wykonania programu. Wyróżniamy złożoność obliczeniową, 
która odnosi się do liczby operacji, oraz złożoność pamięciową, określającą wymaganą ilość pamięci operacyjnej. Aby wyznaczyć złożoność obliczeniową, liczymy operacje wykonywane 
przez algorytm, takie jak inicjalizacja zmiennej czy wykonanie działania. Używamy funkcji $f(n)$, która zwraca ilość operacji w zależności od ilości danych $n$. \newline 
W praktyce stosuje się oszacowania złożoności, korzystając z notacji $O$ (wielkiego O), $\Omega$ (omegi), i $\Theta$ (theta), które pozwalają na górne, dolne lub dokładne oszacowanie 
złożoności. Notacja $O$ koncentruje się na najistotniejszym wyrazie funkcji, pomijając współczynniki, podczas gdy $\Omega$ i $\Theta$ służą odpowiednio do dolnego i dokładnego oszacowania. 
Algorytmy rekomendacji mogą mieć różne złożoności, od stałej $O(1)$, przez liniową $O(n)$, kwadratową $O(n^2)$, aż po logarytmiczną $O(\log n)$ i pierwiastkową $O(\sqrt{n})$, 
co wpływa na ich wydajność przy przetwarzaniu dużych ilości danych.

\section{Wnioski}

Algorytmy rekomendacyjne i techniki z nimi związane, takie jak analiza koszykowa, są niezbędne dla funkcjonowania współczesnego świata cyfrowego, przyczyniając się do personalizacji 
doświadczeń online. Wraz z rozwojem technologii, te algorytmy stają się coraz bardziej zaawansowane, co pozwala na jeszcze lepsze dopasowanie rekomendacji do indywidualnych preferencji 
użytkowników. Przekłada się to na zwiększone zaangażowanie użytkowników i optymalizację sprzedaży na różnych platformach, od e-commerce po muzykę cyfrową i gry komputerowe.

\section{Referencje}
\begin{itemize}
    \item \url{https://twojadrogasukcesu.pl/po-co-nam-algorytm/}
    \item \url{https://redsms.pl/jak-algorytmy-rekomendacyjne-wplywaja-na-nasze-wybory-w-sieci/}
    \item \url{https://cyrekdigital.com/pl/baza-wiedzy/analiza-koszykowa/}
    \item \url{http://bks.uniwersytetradom.pl/publikacje/SI%20w%20grach.pdf}
    \item \href{https://blog.consdata.tech/2018/08/07/algorytmy-rekomendacyjne-przyklad-implementacji-w-pythonie.html}{https://blog.consdata.tech/2018/08/07/algorytmy-rekomendacyjne-przyklad-implementacji-w-pythonie.html}
    \item \url{https://devszczepaniak.pl/zlozonosc-obliczeniowa-algorytmow/}
    \item \url{https://binarnie.pl/zlozonosc-obliczeniowa/?utm_content=cmp-true}
\end{itemize}

\end{document}